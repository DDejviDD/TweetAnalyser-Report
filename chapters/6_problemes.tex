Notre projet est maintenant termin� et nous avons r�sp�ct� nos esp�rance. Toutefois, il reste quelque probl�me pour lesquels on aurait voulu avoir plus de temps afin de les corriger.

\paragraph{L'API de Twitter}
Comme nous l'avons d�j� dit auparavant, nous avons eu beaucoup de soucis avec celle-ci. Il faut savoir que pour utiliser l'API de Twitter il faut poss�der un compte d�veloppeur afin de g�n�rer des tokens pour ses applications. Dans notre cas, nous avons pu profiter des acc�s d'un compte d�velopper �tudiant gratuit, cependant ceci n'est pas suffisant si l'on souhaite passer le projet en phase de production. En effet, les limitations sont telle que l'application serait compl�tement inutilisable.

\paragraph{Garantie des r�sultats}
Actuellement, il nous est impossible de garantir de pouvoir lister toutes les r�ponses d'un tweets. Ceci est li� aux limitations que l'on subit par l'API de Twitter. Nous avons essay� de faire en sorte de minimiser ce probl�me autant que possible.

\paragraph{L'int�grit� des donn�es}
Pour une raison que nous n'avons pas encore r�ussi � trouver, nous nous sommes rendu compte qu'il �tait possible, et ce relativement facilement, de supprimer une analyse de Tweet depuis postman par exemple n'ayant pas le bon token ou, pire encore, sans avoir besoin d'utiliser un token d'authentification. Il semblerait qu'une condition est ignor�e ou pas prise en compte, ceci peut s'expliquer par une mauvaise gestion de la concurrence, une �tape du processus de v�rification qui a �t� oubli�e ou mal ex�cut�e, si l'on avait plus de temps pour le projet, ce probl�me serait le premier que l'on aurait essay� de r�soudre.
