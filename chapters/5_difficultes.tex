Dans cette parties, nous allons lister les quelques probl�me que nous avons rencontr� lors de la r�alisation de ce projet.

\paragraph{L'API de Twitter} 
Toutes API qui existe poss�de des limitations, que ce soit en nombre de requ�tes, en d�bit, un contr�le de saisie faible ou inexistant, des d�lais d'attente bien trop lent, et bien d'autres encore. L'API de Twitter ne fait pas exception � la r�gle, entre les limitations impos�e et sa documentation �parse, nous avons bien failli changer l'objectif du projet.

\paragraph{Programmation concurrente} 
Nous ne pensions pas que la concurrence nous poserait autant de soucis lorsque l'on a commenc� le projet. Nous avons fini par r�soudre une grande majorit� d'entre eux mais il et possible que certains soient rest� ou que nous ne les avons pas d�couvert lors de nos tests.

\paragraph{Le temps} 
Malgr� l'int�r�t port� au projet, une journ�e ne dure que 24 heures. Les fins de semestres sont souvent charg�e voire m�me tr�s charg�e. Parfois, il a �t� difficile de trouver le temps de travailler sur le projet. Sans compter le fait qu'il arrive parfois que l'on se retrouve bloqu� par des bugs pendant plusieurs heures ce qui nous a fait perdre un temps pr�cieux.

\paragraph{Le stress, la pression, et l'envie de bien faire}
Notre c�t� perfectionniste est un avantage mais aussi un d�faut. Afin de cr�er une application dont on peut �tre fier, nous nous sommes mis une pression suppl�mentaire qui aurait pu �tre �vit�e. Mais il est toujours plus plaisant de rendre un beau projet qui fonctionne comme on l'avait souhait�.


