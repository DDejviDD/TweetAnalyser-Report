Cr�� le 21 mars 2006 et lanc� en juillet de la m�me ann�e, Twitter est une application permettant � ses utilisateurs de publier gratuitement des messages de 280 caract�res, �galement appel� tweets, sur internet dont la dur�e de vie est, en moyenne, d'une heure. En 2018, Twitter avait plus de 300 millions d'utilisateurs actif par mois et plus de 500 millions de tweets envoy�s par jour.

Actuellement, ce sont les personnalit�s publique qui envoient le plus de tweets. On retrouve principalement des chefs d'�tats, des artiste, mais aussi des chefs d'entreprise, des entreprises, et des cha�nes de t�l�vision. Twitter n'est d�sormais plus utilis� uniquement pour envoy� de simples messages mais aussi en tant que plateforme d'information et d'�changes lors de certains �v�nements comme par exemple l'Eurovision qui a fait couler beaucoup de bits. Aussi, il existe d�j� de nombreuses entreprises qui proposent un service de support passant par ce m�dium. 

Une entreprise qui publie �norm�ment aura forc�ment besoin de conna�tre l'avis g�n�ral et la r�action de ses abonn�s et autres utilisateurs par rapport � leurs tweets. Il lui serait �galement int�ressant de conna�tre la mani�re dont sont accueilli leurs nouveaux produits mais aussi les r�actions aux tweets des membres de l'entreprises.